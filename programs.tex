\documentclass[]{article}
\usepackage[toc,page]{appendix}
\usepackage{titlesec}
\titlespacing\section{0pt}{12pt plus 4pt minus 2pt}{0pt plus 2pt minus 2pt}
\titlespacing\subsection{0pt}{12pt plus 4pt minus 2pt}{0pt plus 2pt minus 2pt}
\titlespacing\subsubsection{0pt}{12pt plus 4pt minus 2pt}{0pt plus 2pt minus 2pt}
%\titlespacing\subsection{0pt}{12pt plus 4pt minus 2pt}{0pt plus 2pt minus 2pt}

%% Selectively comment out sections that you want to be left out but
%% maintaining the page numbers and other \ref
%\includeonly{%
%  introduction,
%  ref,
%  conclusion
%}

%%% Some commonly used packages (make sure your LaTeX installation
%%% contains these packages, if not ask your senior to help installing
%%% the packages)
%\usepackage{pdfpages}
\usepackage{mathtools}
\usepackage{program}
\usepackage{algorithm}
\usepackage[noend]{algpseudocode}
\usepackage{booktabs}
\usepackage{fixltx2e}
\usepackage{url}
\usepackage{pbox}
\usepackage{graphicx}
\usepackage[T1]{fontenc}
\usepackage{enumitem}
\usepackage{amsmath}
%\usepackage{enumerate} %lower-case roman numerals in enumerate lists
\usepackage{multirow} %for table
\usepackage{hhline} %for table
\usepackage{subcaption} %for subcaption in sub-figures
\setlist{nosep}
\graphicspath{{images/}}
\newcommand{\dud}[1]{\underline{#1}}
%python
\usepackage{listings}
\usepackage{color}

\definecolor{codegreen}{rgb}{0,0.6,0}
\definecolor{codegray}{rgb}{0.5,0.5,0.5}
\definecolor{codepurple}{rgb}{0.58,0,0.82}
\definecolor{backcolour}{rgb}{0.95,0.95,0.92}

\lstdefinestyle{mystyle}{
	backgroundcolor=\color{backcolour},   
	commentstyle=\color{codegreen},
	keywordstyle=\color{magenta},
	numberstyle=\tiny\color{codegray},
	stringstyle=\color{codepurple},
	basicstyle=\footnotesize,
	breakatwhitespace=false,         
	breaklines=true,                 
	captionpos=b,                    
	keepspaces=true,                 
	numbers=left,                    
	numbersep=5pt,                  
	showspaces=false,                
	showstringspaces=false,
	showtabs=false,                  
	tabsize=2
}

\lstset{style=mystyle}
%python2 
% Default fixed font does not support bold face
\DeclareFixedFont{\ttb}{T1}{txtt}{bx}{n}{12} % for bold
\DeclareFixedFont{\ttm}{T1}{txtt}{m}{n}{12}  % for normal

% Custom colors
%\usepackage{color}
\definecolor{deepblue}{rgb}{0,0,0.5}
\definecolor{deepred}{rgb}{0.6,0,0}
\definecolor{deepgreen}{rgb}{0,0.5,0}

%\usepackage{listings}
% Python style for highlighting
\newcommand\pythonstyle{\lstset{
		language=Python,
		basicstyle=\ttm,
		otherkeywords={self,if,while,log},             % Add keywords here
		keywordstyle=\ttb\color{deepblue},
		emph={MyClass,__init__},          % Custom highlighting
		emphstyle=\ttb\color{deepred},    % Custom highlighting style
		stringstyle=\color{deepgreen},
		frame=tb,                         % Any extra options here
		showstringspaces=false            % 
	}}
	%endpython
	%\renewcommand{\chaptername}{}
	\usepackage{titlesec}


%opening

\begin{document}
\begin{lstlisting}[language=Python, caption=Python version of copy1 example in \cite{denning}. goal: information flow from x to y, label={lst:copy1} ]
#Procedure copy1
x = 0
z = 0
y = 0
if x == 0:
z=1
if z == 0:
y=1
\end{lstlisting}
\begin{lstlisting}[language=Python, caption=Python version of copy2 example in \cite{denning}. goal: information flow from x to y, label={lst:copy2} ]
# Procedure copy2
x = 0
z = 1
y = -1
while z == 1:
y = y + 1
if y == 0:
z = x
else:
z = 0
\end{lstlisting}


\begin{lstlisting}[language=Python, caption=Python version of copy3 example in \cite{denning}. goal: information flow from x to y, label={lst:copy3} ]
#copy3 synchronization flow
import thread
import time
import threading

#x=7
#y=6
#def copy3(x,y):     # copy x to y
s0 = threading.Event()
s1 = threading.Event()

def thread1():
global x
if x==0:
s0.set()
else:
s1.set()

def thread2():
global y
s0.wait()
s0.clear()
y=1
s1.set()

def thread3():
global y
s1.wait()
s1.clear()
y=0
s0.set()

try:
thread.start_new_thread(thread1,())
thread.start_new_thread(thread2,())
thread.start_new_thread(thread3,())
except:
print "Error: unable to start thread"

\end{lstlisting}


\begin{lstlisting}[language=Python, caption=Python version of copy4 example in \cite{denning}. goal: information flow from x to y, label={lst:copy4} ]
#Procedure copy4
import thread
import time
import threading

def thread1():
global x
global e0
global e1
if x==0:
e0 = False
else:
e1 = False

def thread2(): 
global e0
global e1
global y
while e0
pass
y = 1
e1 = False

def thread3():
global e1
global e0
global y
while e1:
pass
y = 0
e0 = False

thread.start_new_thread(thread1,())
thread.start_new_thread(thread2,())
thread.start_new_thread(thread3,())\end{lstlisting}

\begin{lstlisting}[language=Python, caption=Python version of copy5 example in \cite{denning}. goal: information flow from x to y, label={lst:copy5} ]
#Procedure copy5
y = 0
while x==0 :
pass
y = 1
\end{lstlisting}
\begin{lstlisting}[language=Python, caption=Python version of copy6 example in \cite{denning}. goal: information flow from x to y, label={lst:copy6} ]
#copy6
z = 0
ssum = 0
y = 0
while z == 0 :
	ssum = ssum + x       
	y = y + 1
\end{lstlisting}
\begin{lstlisting}[language=Python, caption=Python version of dynamic label example in \cite{denning}. goal: information flow from x to y, label={lst:dynamic} ]
def fun(x, y, z):
	a = x
	y = a
	a = z
fun(x, y, z)
\end{lstlisting}
\end{document}
