\chapter{Introduction}
\section{Language Based Security}
Information security is major concern nowadays. Systems are vulnerable to various attacks and exposed to threats via network. There are many approaches to enforce security policies in the system one of them is language based security, this approach focuses on enforcement of security policies on a application using program analysis. There are three main branches in language based security (a) Reference Monitor (b) Type Safe (c) Certifying Compiler this report focuses on certifying compiler, certifying compiler approach is based on program analysis compiler checks whether program follows security policy. Compiler does certification from outside of the system so it matches with principle of security model (i) Principle of least privilege (ii) Minimal computing base \cite{lang}. Static analysis is simple for this type of certification because whole process completes in a one go. Dynamic analysis becomes necessary if information flow occurs only at run time. So hybrid approach can cover all type of ananlysis.    
\section{Noninterference}
One of the earliest formal work in information security is concept of noninterference created by Goguen and Meseguer \cite{noni} in the context of MLS(Multi Level Secure). It gave concept to determine information leak in system of multiple users, suppose that there are two group of users A and B in a system S and if user of group A interacts with System S then view of users of B remains unchanged. This concept was developed for deterministic systems. Due to support for nondeterminism in most of programming languages, researcher \cite{ques} questioned relevance of noninterference for security of program. 
Volpano et al \cite{volpano} says that noninterference can be used for current day programming languages by using purely value based interpretation of noninterference, and with the help of Denning's certification semantics. Volpano's work regarding noninterference has set standards in language based security.          
\section{Security of a program}
There are various techniques to ensure security of program, information flow control is one of them. 

\subsection{Motivation}
In the field of data security, there are a lot of approaches to prevent leak of information for example cryptography takes care of confidentiality and integrity while data is transmitting through less secure networks, access permission on files prevent unauthorized access to files in a system where users have different privileges. But at the time of execution of program, data used in the program is vulnerable to various attacks so to maintain security at the time of execution of program and processing of data, information flow policies are used. The subject is defined as an executing authority it can be a user or parent process, object can be a file, program variable, memory location etc. In a multilevel security system a subject has permission related to objects. Information flow verification of program only considering objects may seem to be secure but with a particular subject same program may be insecure, so we considered subjects in static analysis of python program.   

\subsection{Goals}
To develop a platform that takes input a python program and labels of each variable used in the program, and provide answers to various queries regarding the security of information flow.   
